%% Generated by Sphinx.
\def\sphinxdocclass{report}
\documentclass[letterpaper,10pt,english]{sphinxmanual}
\ifdefined\pdfpxdimen
   \let\sphinxpxdimen\pdfpxdimen\else\newdimen\sphinxpxdimen
\fi \sphinxpxdimen=.75bp\relax

\PassOptionsToPackage{warn}{textcomp}
\usepackage[utf8]{inputenc}
\ifdefined\DeclareUnicodeCharacter
% support both utf8 and utf8x syntaxes
  \ifdefined\DeclareUnicodeCharacterAsOptional
    \def\sphinxDUC#1{\DeclareUnicodeCharacter{"#1}}
  \else
    \let\sphinxDUC\DeclareUnicodeCharacter
  \fi
  \sphinxDUC{00A0}{\nobreakspace}
  \sphinxDUC{2500}{\sphinxunichar{2500}}
  \sphinxDUC{2502}{\sphinxunichar{2502}}
  \sphinxDUC{2514}{\sphinxunichar{2514}}
  \sphinxDUC{251C}{\sphinxunichar{251C}}
  \sphinxDUC{2572}{\textbackslash}
\fi
\usepackage{cmap}
\usepackage[T1]{fontenc}
\usepackage{amsmath,amssymb,amstext}
\usepackage{babel}



\usepackage{times}
\expandafter\ifx\csname T@LGR\endcsname\relax
\else
% LGR was declared as font encoding
  \substitutefont{LGR}{\rmdefault}{cmr}
  \substitutefont{LGR}{\sfdefault}{cmss}
  \substitutefont{LGR}{\ttdefault}{cmtt}
\fi
\expandafter\ifx\csname T@X2\endcsname\relax
  \expandafter\ifx\csname T@T2A\endcsname\relax
  \else
  % T2A was declared as font encoding
    \substitutefont{T2A}{\rmdefault}{cmr}
    \substitutefont{T2A}{\sfdefault}{cmss}
    \substitutefont{T2A}{\ttdefault}{cmtt}
  \fi
\else
% X2 was declared as font encoding
  \substitutefont{X2}{\rmdefault}{cmr}
  \substitutefont{X2}{\sfdefault}{cmss}
  \substitutefont{X2}{\ttdefault}{cmtt}
\fi


\usepackage[Bjarne]{fncychap}
\usepackage[,numfigreset=1,mathnumfig]{sphinx}

\fvset{fontsize=\small}
\usepackage{geometry}


% Include hyperref last.
\usepackage{hyperref}
% Fix anchor placement for figures with captions.
\usepackage{hypcap}% it must be loaded after hyperref.
% Set up styles of URL: it should be placed after hyperref.
\urlstyle{same}

\addto\captionsenglish{\renewcommand{\contentsname}{Main docs}}

\usepackage{sphinxmessages}
\setcounter{tocdepth}{1}



\title{CoFEA Initiative}
\date{Nov 23, 2020}
\release{}
\author{BlogTechniczny.pl}
\newcommand{\sphinxlogo}{\vbox{}}
\renewcommand{\releasename}{}
\makeindex
\begin{document}

\pagestyle{empty}
\sphinxmaketitle
\pagestyle{plain}
\sphinxtableofcontents
\pagestyle{normal}
\phantomsection\label{\detokenize{index::doc}}
\begin{figure}[htbp]
\centering

\noindent\sphinxincludegraphics[width=400\sphinxpxdimen]{{cofea-logo}.png}
\end{figure}



CoFEA is an initiative that aims to bring open\sphinxhyphen{}source simulation software closer to industry standards. Even though CoFEA itself sounds like the magic drink which drives engineering minds, it is a combination of the words \sphinxstyleemphasis{cooperation} and \sphinxstyleemphasis{Finite Element Analysis}. That truly describes the goals of this project which are:
\begin{itemize}
\item {} 
raise awareness of available open\sphinxhyphen{}source simulation software

\item {} 
increase confidence in using this software by testing it with benchmarks and real\sphinxhyphen{}life problems

\item {} 
develop Python tools and scripts to make usage easier and simpler

\item {} 
gather information on how to set up the simulation environment correctly

\end{itemize}


\chapter{Site contents}
\label{\detokenize{index:site-contents}}

\section{Simulation environment setup}
\label{\detokenize{software_setup/index:simulation-environment-setup}}\label{\detokenize{software_setup/index::doc}}
This section describes how to set up simulation software


\subsection{Calculix}
\label{\detokenize{software_setup/calculix:calculix}}\label{\detokenize{software_setup/calculix::doc}}
To use Calculix in multithreading mode it is needed to compile from source. In order to do so it is necessary to compile ARPACK and SPOOLES library nad install the required tools. The code which is presented here is already changed and ready for compilation. Please read the original documentation first from http://www.dhondt.de/ccx\_2.17.README.INSTALL


\subsubsection{Tools}
\label{\detokenize{software_setup/calculix:tools}}\begin{enumerate}
\sphinxsetlistlabels{\arabic}{enumi}{enumii}{}{.}%
\item {} 
Install the required tools for Calculix:
\begin{itemize}
\item {} 
gfortran

\item {} 
make

\item {} 
f2c

\item {} 
liblapack3

\item {} 
liblapack\sphinxhyphen{}dev

\item {} 
libexodusii\sphinxhyphen{}dev

\item {} 
libgl1\sphinxhyphen{}mesa\sphinxhyphen{}dev

\item {} 
libglu1\sphinxhyphen{}mesa\sphinxhyphen{}dev

\item {} 
libxi\sphinxhyphen{}dev

\item {} 
libxmu\sphinxhyphen{}dev

\item {} 
gcc

\end{itemize}

\end{enumerate}

using your package manager. For Ubuntu and Debian\sphinxhyphen{}oriented system the command should be like:

\begin{sphinxVerbatim}[commandchars=\\\{\}]
sudo apt\PYGZhy{}get install gfortran make f2c liblapack3 liblapack\PYGZhy{}dev libexodusii\PYGZhy{}dev libgl1\PYGZhy{}mesa\PYGZhy{}dev libglu1\PYGZhy{}mesa\PYGZhy{}dev libxi\PYGZhy{}dev libxmu\PYGZhy{}dev
\end{sphinxVerbatim}


\subsubsection{SPOOLES}
\label{\detokenize{software_setup/calculix:spooles}}
SPOOLES library should be obtained from this site http://www.netlib.org/linalg/spooles/spooles.2.2.tgz using wget command.

\begin{sphinxVerbatim}[commandchars=\\\{\}]
wget http://www.netlib.org/linalg/spooles/spooles.2.2.tgz
\end{sphinxVerbatim}

After downloading the file, it is mandatory to create folder SPOOLES.2.2. move the archive there and unpack it. It can be done with the following commands:

\begin{sphinxVerbatim}[commandchars=\\\{\}]
mkdir SPOOLES.2.2
mv spooles.2.2.tgz SPOOLES.2.2
cd SPOOLES.2.2
tar xvf spooles.2.2.tgz
\end{sphinxVerbatim}

Change directory to SPOOLES.2.2 with cd:

\begin{sphinxVerbatim}[commandchars=\\\{\}]
cd SPOOLES.2.2
\end{sphinxVerbatim}

Then  uncomment 14 line comment line 15 in Make.inc file with your text editor. It can be done by:

\begin{sphinxVerbatim}[commandchars=\\\{\}]
gedit Make.inc
\end{sphinxVerbatim}

The file after changes should look like:

\begin{sphinxVerbatim}[commandchars=\\\{\}]
CC = gcc
\PYGZsh{}  CC = /usr/lang\PYGZhy{}4.0/bin/cc
\end{sphinxVerbatim}

Now the files are ready for compilation. Use:

\begin{sphinxVerbatim}[commandchars=\\\{\}]
make lib  
\end{sphinxVerbatim}

Then it is needed to compile the MT library. It is done with following commands:

\begin{sphinxVerbatim}[commandchars=\\\{\}]
cd MT/src/
make   
\end{sphinxVerbatim}


\subsubsection{ARPACK}
\label{\detokenize{software_setup/calculix:arpack}}
Next step is to compile ARPACK library. It is needed to download 2 seperate archives. Obtain it from https://www.caam.rice.edu/software/ARPACK/SRC/arpack96.tar.gz and https://www.caam.rice.edu/software/ARPACK/SRC/patch.tar.gz

\begin{sphinxVerbatim}[commandchars=\\\{\}]
wget https://www.caam.rice.edu/software/ARPACK/SRC/arpack96.tar.gz
wget https://www.caam.rice.edu/software/ARPACK/SRC/patch.tar.gz
\end{sphinxVerbatim}

Then unpack files with tar commands

\begin{sphinxVerbatim}[commandchars=\\\{\}]
tar xvf arpack96.tar.gz
tar xvf patch.tar.gz
\end{sphinxVerbatim}
\begin{itemize}
\item {} 
In \sphinxcode{\sphinxupquote{ARPACK/ARmake.inc}} change:
\begin{itemize}
\item {} 
\sphinxcode{\sphinxupquote{home = \$(HOME)/ARPACK}} to your ARPACK directory

\item {} 
\sphinxcode{\sphinxupquote{PLAT = SUN4}} to \sphinxcode{\sphinxupquote{PLAT = linux}}

\item {} 
\sphinxcode{\sphinxupquote{FC = f77}} to \sphinxcode{\sphinxupquote{FC = gfortran}}

\item {} 
\sphinxcode{\sphinxupquote{FFLAGS = \sphinxhyphen{}O \sphinxhyphen{}cg89}} to \sphinxcode{\sphinxupquote{FFLAGS = \sphinxhyphen{}O2}}

\item {} 
\sphinxcode{\sphinxupquote{MAKE = /bin/make}} to \sphinxcode{\sphinxupquote{MAKE = make}}

\item {} 
\sphinxcode{\sphinxupquote{SHELL = /bin/sh}} to \sphinxcode{\sphinxupquote{SHELL = shell}}

\end{itemize}

\item {} 
In \sphinxcode{\sphinxupquote{ARPACK/UTIL/second.f}} change: \sphinxcode{\sphinxupquote{EXTERNAL ETIME}} to \sphinxcode{\sphinxupquote{*EXTERNAL ETIME}}

\end{itemize}

Then mowe to ARPACK directory and run:

\begin{sphinxVerbatim}[commandchars=\\\{\}]
make lib   
\end{sphinxVerbatim}


\subsubsection{Calculix compilation}
\label{\detokenize{software_setup/calculix:calculix-compilation}}
Obtain Calculix source code from: http://www.dhondt.de/ccx\_2.17.src.tar.bz2 using wget to your Calculix directory.

\begin{sphinxVerbatim}[commandchars=\\\{\}]
wget http://www.dhondt.de/ccx\PYGZus{}2.17.src.tar.bz2
\end{sphinxVerbatim}

To unpack archieve use:

\begin{sphinxVerbatim}[commandchars=\\\{\}]
bunzip2 ccx\PYGZus{}2.17.src.tar.bz2
tar xvf ccx\PYGZus{}2.17.src.tar
\end{sphinxVerbatim}

In order to compile Calculix in multithreading mode it is needed to change the Makefile from singlethread to multithread one. It can be simply done by:

\begin{sphinxVerbatim}[commandchars=\\\{\}]
cd CalculiX/ccx\PYGZus{}2.17/src
mv Makefile Makefile\PYGZus{}ST
mv Makefile\PYGZus{}MT Makefile
\end{sphinxVerbatim}
\begin{itemize}
\item {} 
In \sphinxcode{\sphinxupquote{Makefile}} change:
\begin{itemize}
\item {} 
\sphinxcode{\sphinxupquote{CC=cc}} to \sphinxcode{\sphinxupquote{CC=gcc}}

\item {} 
in 15 line   \sphinxcode{\sphinxupquote{../../../ARPACK/libarpack\_INTEL.a \textbackslash{}}} to \sphinxcode{\sphinxupquote{../../../ARPACK/libarpack\_linux.a \textbackslash{}}}

\end{itemize}

\end{itemize}

Then it is possible to compile Calculix with:

\begin{sphinxVerbatim}[commandchars=\\\{\}]
make
\end{sphinxVerbatim}

Happy meshing!


\subsection{Code\_Aster Compilation}
\label{\detokenize{software_setup/code_aster:code-aster-compilation}}\label{\detokenize{software_setup/code_aster::doc}}
To use Code\_aster in multithreading mode on UBUNTU 18.04 it is needed to compile from source. In order to do so it is necessary to install required dependencies tools. The source code of Code\_aster can be obtained from this repository or https://www.code\sphinxhyphen{}aster.org/spip.php?rubrique21


\subsubsection{Prerequisities}
\label{\detokenize{software_setup/code_aster:prerequisities}}\begin{enumerate}
\sphinxsetlistlabels{\arabic}{enumi}{enumii}{}{.}%
\item {} 
Install the required tools for Code\_aster:
\begin{itemize}
\item {} 
gcc, g++, gfortran,

\item {} 
cmake,

\item {} 
python3,

\item {} 
python3\sphinxhyphen{}dev,

\item {} 
python3\sphinxhyphen{}numpy,

\item {} 
tk,

\item {} 
bison,,

\item {} 
flex,

\item {} 
liblapack\sphinxhyphen{}dev, libblas\sphinxhyphen{}dev ou libopenblas\sphinxhyphen{}dev,

\item {} 
libboost\sphinxhyphen{}python\sphinxhyphen{}dev (+ libboost\sphinxhyphen{}numpy\sphinxhyphen{}dev on ubuntu, boost\sphinxhyphen{}devel on centos),

\item {} 
zlib (named zlib1g\sphinxhyphen{}dev sur debian/ubuntu).

\end{itemize}

\end{enumerate}

using your package manager. For Ubuntu and Debian\sphinxhyphen{}oriented system the command should be like:

\begin{sphinxVerbatim}[commandchars=\\\{\}]
sudo apt\PYGZhy{}get install gfortran
\end{sphinxVerbatim}


\subsubsection{Compilation}
\label{\detokenize{software_setup/code_aster:compilation}}
In order to compile Code\_aster please download the source code to the installation directory.

\begin{sphinxVerbatim}[commandchars=\\\{\}]
wget https://www.code\PYGZhy{}aster.org/FICHIERS/aster\PYGZhy{}full\PYGZhy{}src\PYGZhy{}14.6.0\PYGZhy{}1.noarch.tar.gz
\end{sphinxVerbatim}

After downloading the file, please unpack it using the following commands:

\begin{sphinxVerbatim}[commandchars=\\\{\}]
gunzip aster\PYGZhy{}full\PYGZhy{}src\PYGZhy{}14.6.0\PYGZhy{}1.noarch.tar.gz
tar \PYGZhy{}xvf aster\PYGZhy{}full\PYGZhy{}src\PYGZhy{}14.6.0\PYGZhy{}1.noarch.tar
\end{sphinxVerbatim}

To compile the code. please use command:

\begin{sphinxVerbatim}[commandchars=\\\{\}]
python3 setup.py install \PYGZhy{}\PYGZhy{}prefix=/your/installation/path/to/code/aster
\end{sphinxVerbatim}

To be able to use Code\_aster and ASTK in every directory on your system you need to add line to .bashrc file

\begin{sphinxVerbatim}[commandchars=\\\{\}]
cd
source /your/installation/path/to/code/aster/etc/codeaster/profile.sh \PYGZgt{} .bashrc
source .bashrc
\end{sphinxVerbatim}

Sometimes this operation need root privileges so add sudo at the beginnings of the commands.


\subsubsection{How to use ASTK with Code\_aster}
\label{\detokenize{software_setup/code_aster:how-to-use-astk-with-code-aster}}
Change your directory to the simulation directory. In this directory there should be solver input file (*.comm) and mesh file  Create two dummy files result.rmed (there will be the results of the simulation) and error.mess (logfile of the solver) and .astk file for ASTK enviroment. Use touch command:

\begin{sphinxVerbatim}[commandchars=\\\{\}]
touch result.rmed
touch error.mess
touch RunCase\PYGZus{}1.astk
\end{sphinxVerbatim}

Type the

\begin{sphinxVerbatim}[commandchars=\\\{\}]
astk
\end{sphinxVerbatim}

in the terminal. The ASTK Window should appear with the job progress window.

\begin{figure}[htbp]
\centering

\noindent\sphinxincludegraphics[width=450\sphinxpxdimen]{{first}.png}
\end{figure}

In the Base path select your working directory, then add *.comm file, mesh file, *.rmed and *.mess to run the simulation with the blue, open folder icon. To start the simulation press run.

\begin{figure}[htbp]
\centering

\noindent\sphinxincludegraphics[width=450\sphinxpxdimen]{{second}.png}
\end{figure}

Here’s a sample post list:
\begin{itemize}
\end{itemize}


\section{CoFEA mesh converter}
\label{\detokenize{pyCofea/index:cofea-mesh-converter}}\label{\detokenize{pyCofea/index::doc}}
This section describes how to set up simulation software


\subsection{About CoFEA}
\label{\detokenize{pyCofea/about:about-cofea}}\label{\detokenize{pyCofea/about::doc}}
The idea behind the CoFEA module is to allow converting models data between different simulation environments.


\subsection{First steps with CoFEA}
\label{\detokenize{pyCofea/about:first-steps-with-cofea}}
The CoFEA module is written in a way that allows to either create a mesh from scratch or import it from different source. Here is the example how to prepare a mesh which is then exported to Calculix input deck. It can also be exported to .db file from which it can be imported back.

\begin{sphinxVerbatim}[commandchars=\\\{\}]
\PYG{g+gp}{\PYGZgt{}\PYGZgt{}\PYGZgt{} }\PYG{c+c1}{\PYGZsh{} example: prepare db file from scratch}
\PYG{g+gp}{\PYGZgt{}\PYGZgt{}\PYGZgt{} }\PYG{c+c1}{\PYGZsh{} prepare nodes}
\PYG{g+gp}{\PYGZgt{}\PYGZgt{}\PYGZgt{} }\PYG{n}{n1} \PYG{o}{=} \PYG{n}{cofea}\PYG{o}{.}\PYG{n}{Node}\PYG{p}{(}\PYG{n}{nLabel}\PYG{o}{=}\PYG{l+m+mi}{1}\PYG{p}{,} \PYG{n}{nCoords}\PYG{o}{=}\PYG{p}{(}\PYG{l+m+mf}{1.0}\PYG{p}{,} \PYG{l+m+mf}{0.0}\PYG{p}{,} \PYG{l+m+mf}{0.0}\PYG{p}{)}\PYG{p}{)}
\PYG{g+gp}{\PYGZgt{}\PYGZgt{}\PYGZgt{} }\PYG{n}{n2} \PYG{o}{=} \PYG{n}{cofea}\PYG{o}{.}\PYG{n}{Node}\PYG{p}{(}\PYG{n}{nLabel}\PYG{o}{=}\PYG{l+m+mi}{2}\PYG{p}{,} \PYG{n}{nCoords}\PYG{o}{=}\PYG{p}{(}\PYG{l+m+mf}{0.0}\PYG{p}{,} \PYG{l+m+mf}{1.0}\PYG{p}{,} \PYG{l+m+mf}{0.0}\PYG{p}{)}\PYG{p}{)}
\PYG{g+gp}{\PYGZgt{}\PYGZgt{}\PYGZgt{} }\PYG{n}{n3} \PYG{o}{=} \PYG{n}{cofea}\PYG{o}{.}\PYG{n}{Node}\PYG{p}{(}\PYG{n}{nLabel}\PYG{o}{=}\PYG{l+m+mi}{3}\PYG{p}{,} \PYG{n}{nCoords}\PYG{o}{=}\PYG{p}{(}\PYG{l+m+mf}{0.0}\PYG{p}{,} \PYG{l+m+mf}{0.0}\PYG{p}{,} \PYG{l+m+mf}{1.0}\PYG{p}{)}\PYG{p}{)}
\PYG{g+gp}{\PYGZgt{}\PYGZgt{}\PYGZgt{} }\PYG{c+c1}{\PYGZsh{} put nodes into the list}
\PYG{g+gp}{\PYGZgt{}\PYGZgt{}\PYGZgt{} }\PYG{n}{nodeList} \PYG{o}{=} \PYG{p}{[}\PYG{n}{n1}\PYG{p}{,} \PYG{n}{n2}\PYG{p}{,} \PYG{n}{n3}\PYG{p}{]}
\PYG{g+gp}{\PYGZgt{}\PYGZgt{}\PYGZgt{} }\PYG{c+c1}{\PYGZsh{} prepare elements}
\PYG{g+gp}{\PYGZgt{}\PYGZgt{}\PYGZgt{} }\PYG{n}{e1} \PYG{o}{=} \PYG{n}{cofea}\PYG{o}{.}\PYG{n}{Element}\PYG{p}{(}\PYG{n}{elType}\PYG{o}{=}\PYG{l+s+s1}{\PYGZsq{}}\PYG{l+s+s1}{C3D4}\PYG{l+s+s1}{\PYGZsq{}}\PYG{p}{,} \PYG{n}{elLabel}\PYG{o}{=}\PYG{l+m+mi}{1}\PYG{p}{,}
\PYG{g+gp}{\PYGZgt{}\PYGZgt{}\PYGZgt{} }               \PYG{n}{elConnect}\PYG{o}{=}\PYG{p}{(}\PYG{l+m+mi}{0}\PYG{p}{,} \PYG{l+m+mi}{1}\PYG{p}{,} \PYG{l+m+mi}{2}\PYG{p}{)}\PYG{p}{,}
\PYG{g+gp}{\PYGZgt{}\PYGZgt{}\PYGZgt{} }               \PYG{n}{partAllNodes}\PYG{o}{=}\PYG{n}{nodeList}\PYG{p}{)}
\PYG{g+gp}{\PYGZgt{}\PYGZgt{}\PYGZgt{} }\PYG{c+c1}{\PYGZsh{} put elements into the list}
\PYG{g+gp}{\PYGZgt{}\PYGZgt{}\PYGZgt{} }\PYG{n}{elementList} \PYG{o}{=} \PYG{p}{[}\PYG{n}{e1}\PYG{p}{,} \PYG{p}{]}
\PYG{g+gp}{\PYGZgt{}\PYGZgt{}\PYGZgt{} }\PYG{c+c1}{\PYGZsh{} create part from nodes and elements}
\PYG{g+gp}{\PYGZgt{}\PYGZgt{}\PYGZgt{} }\PYG{n}{part} \PYG{o}{=} \PYG{n}{cofea}\PYG{o}{.}\PYG{n}{PartMesh}\PYG{p}{(}\PYG{n}{partName}\PYG{o}{=}\PYG{l+s+s1}{\PYGZsq{}}\PYG{l+s+s1}{TestPart}\PYG{l+s+s1}{\PYGZsq{}}\PYG{p}{,}
\PYG{g+gp}{\PYGZgt{}\PYGZgt{}\PYGZgt{} }                  \PYG{n}{partNodes}\PYG{o}{=}\PYG{n}{nodeList}\PYG{p}{,}
\PYG{g+gp}{\PYGZgt{}\PYGZgt{}\PYGZgt{} }                  \PYG{n}{partElements}\PYG{o}{=}\PYG{n}{elementList}\PYG{p}{)}
\PYG{g+gp}{\PYGZgt{}\PYGZgt{}\PYGZgt{} }\PYG{c+c1}{\PYGZsh{} put parts into the list}
\PYG{g+gp}{\PYGZgt{}\PYGZgt{}\PYGZgt{} }\PYG{n}{partList} \PYG{o}{=} \PYG{p}{[}\PYG{n}{part}\PYG{p}{,} \PYG{p}{]}
\PYG{g+gp}{\PYGZgt{}\PYGZgt{}\PYGZgt{} }\PYG{c+c1}{\PYGZsh{} create a model}
\PYG{g+gp}{\PYGZgt{}\PYGZgt{}\PYGZgt{} }\PYG{n}{model} \PYG{o}{=} \PYG{n}{cofea}\PYG{o}{.}\PYG{n}{ExportMesh}\PYG{p}{(}\PYG{n}{modelName}\PYG{o}{=}\PYG{l+s+s1}{\PYGZsq{}}\PYG{l+s+s1}{test}\PYG{l+s+s1}{\PYGZsq{}}\PYG{p}{,}
\PYG{g+gp}{\PYGZgt{}\PYGZgt{}\PYGZgt{} }                     \PYG{n}{listOfParts}\PYG{o}{=}\PYG{n}{partList}\PYG{p}{)}
\PYG{g+gp}{\PYGZgt{}\PYGZgt{}\PYGZgt{} }\PYG{c+c1}{\PYGZsh{} do some operations with the model}
\PYG{g+gp}{\PYGZgt{}\PYGZgt{}\PYGZgt{} }\PYG{c+c1}{\PYGZsh{} for example export model to calculix file}
\PYG{g+gp}{\PYGZgt{}\PYGZgt{}\PYGZgt{} }\PYG{n}{model}\PYG{o}{.}\PYG{n}{exportToCalculix}\PYG{p}{(}\PYG{n}{exportedFilename}\PYG{o}{=}\PYG{l+s+s1}{\PYGZsq{}}\PYG{l+s+s1}{test.inp}\PYG{l+s+s1}{\PYGZsq{}}\PYG{p}{)}
\PYG{g+gp}{\PYGZgt{}\PYGZgt{}\PYGZgt{} }\PYG{c+c1}{\PYGZsh{} or save it to db file}
\PYG{g+gp}{\PYGZgt{}\PYGZgt{}\PYGZgt{} }\PYG{n}{model}\PYG{o}{.}\PYG{n}{saveToDbFile}\PYG{p}{(}\PYG{l+s+s1}{\PYGZsq{}}\PYG{l+s+s1}{dbFile.db}\PYG{l+s+s1}{\PYGZsq{}}\PYG{p}{)}
\end{sphinxVerbatim}

In order to load the mesh from db file, the following function can be used:

\begin{sphinxVerbatim}[commandchars=\\\{\}]
\PYG{g+gp}{\PYGZgt{}\PYGZgt{}\PYGZgt{} }\PYG{c+c1}{\PYGZsh{} example: load mesh from db file}
\PYG{g+gp}{\PYGZgt{}\PYGZgt{}\PYGZgt{} }\PYG{n}{m} \PYG{o}{=} \PYG{n}{cofea}\PYG{o}{.}\PYG{n}{ExportMesh}\PYG{o}{.}\PYG{n}{importFromDbFile}\PYG{p}{(}\PYG{n}{pathToDbFile}\PYG{o}{=}\PYG{l+s+s1}{\PYGZsq{}}\PYG{l+s+s1}{dbFile.db}\PYG{l+s+s1}{\PYGZsq{}}\PYG{p}{)}
\PYG{g+gp}{\PYGZgt{}\PYGZgt{}\PYGZgt{} }\PYG{n}{m}\PYG{o}{.}\PYG{n}{exportToCalculix}\PYG{p}{(}\PYG{n}{exportedFilename}\PYG{o}{=}\PYG{l+s+s1}{\PYGZsq{}}\PYG{l+s+s1}{test.inp}\PYG{l+s+s1}{\PYGZsq{}}\PYG{p}{)}
\end{sphinxVerbatim}


\subsection{CoFEA module implementation}
\label{\detokenize{pyCofea/api:cofea-module-implementation}}\label{\detokenize{pyCofea/api::doc}}
The following section contains a documentation of the CoFEA module implementation.
\index{Mesh (class in cofea)@\spxentry{Mesh}\spxextra{class in cofea}}

\begin{fulllineitems}
\phantomsection\label{\detokenize{pyCofea/api:cofea.Mesh}}\pysiglinewithargsret{\sphinxbfcode{\sphinxupquote{class }}\sphinxcode{\sphinxupquote{cofea.}}\sphinxbfcode{\sphinxupquote{Mesh}}}{\emph{\DUrole{n}{modelName}}, \emph{\DUrole{n}{listOfParts}}}{}
Mesh is the most general class used to import, store
and export mesh data. The idea is to initialise the constructor,
then import mesh from db file. Finally, the mesh can be
exported to an external format. For example

\begin{sphinxVerbatim}[commandchars=\\\{\}]
\PYG{g+gp}{\PYGZgt{}\PYGZgt{}\PYGZgt{} }\PYG{n}{mesh} \PYG{o}{=} \PYG{n}{Mesh}\PYG{p}{(}\PYG{p}{)}
\PYG{g+gp}{\PYGZgt{}\PYGZgt{}\PYGZgt{} }\PYG{n}{mesh}\PYG{o}{.}\PYG{n}{importFromDbFile}\PYG{p}{(}\PYG{n}{pathToDbFile}\PYG{o}{=}\PYG{l+s+s1}{\PYGZsq{}}\PYG{l+s+s1}{mesh.db}\PYG{l+s+s1}{\PYGZsq{}}\PYG{p}{)}
\PYG{g+gp}{\PYGZgt{}\PYGZgt{}\PYGZgt{} }\PYG{n}{mesh}\PYG{o}{.}\PYG{n}{exportToCalculix}\PYG{p}{(}\PYG{n}{exportedFilename}\PYG{o}{=}\PYG{l+s+s1}{\PYGZsq{}}\PYG{l+s+s1}{Calculix.inp}\PYG{l+s+s1}{\PYGZsq{}}\PYG{p}{)}
\PYG{g+gp}{\PYGZgt{}\PYGZgt{}\PYGZgt{} }\PYG{n}{mesh}\PYG{o}{.}\PYG{n}{exportToUnvFormat}\PYG{p}{(}\PYG{n}{exportedFilename}\PYG{o}{=}\PYG{l+s+s1}{\PYGZsq{}}\PYG{l+s+s1}{Salome.unv}\PYG{l+s+s1}{\PYGZsq{}}\PYG{p}{)}
\end{sphinxVerbatim}
\index{modelName (cofea.Mesh attribute)@\spxentry{modelName}\spxextra{cofea.Mesh attribute}}

\begin{fulllineitems}
\phantomsection\label{\detokenize{pyCofea/api:cofea.Mesh.modelName}}\pysigline{\sphinxbfcode{\sphinxupquote{modelName}}}
name of imported model
\begin{quote}\begin{description}
\item[{Type}] \leavevmode
string

\end{description}\end{quote}

\end{fulllineitems}

\index{parts (cofea.Mesh attribute)@\spxentry{parts}\spxextra{cofea.Mesh attribute}}

\begin{fulllineitems}
\phantomsection\label{\detokenize{pyCofea/api:cofea.Mesh.parts}}\pysigline{\sphinxbfcode{\sphinxupquote{parts}}}
list of imported parts
\begin{quote}\begin{description}
\item[{Type}] \leavevmode
\sphinxhref{https://docs.python.org/3.8/library/stdtypes.html\#list}{list}

\end{description}\end{quote}

\end{fulllineitems}

\index{assemblyMesh (cofea.Mesh attribute)@\spxentry{assemblyMesh}\spxextra{cofea.Mesh attribute}}

\begin{fulllineitems}
\phantomsection\label{\detokenize{pyCofea/api:cofea.Mesh.assemblyMesh}}\pysigline{\sphinxbfcode{\sphinxupquote{assemblyMesh}}}
the dictionary contains parts, but those have renumbered        node and element labels, so that the labels are not        repeating
\begin{quote}\begin{description}
\item[{Type}] \leavevmode
dictionary

\end{description}\end{quote}

\end{fulllineitems}

\index{computeAssemblyMesh() (cofea.Mesh method)@\spxentry{computeAssemblyMesh()}\spxextra{cofea.Mesh method}}

\begin{fulllineitems}
\phantomsection\label{\detokenize{pyCofea/api:cofea.Mesh.computeAssemblyMesh}}\pysiglinewithargsret{\sphinxbfcode{\sphinxupquote{computeAssemblyMesh}}}{}{}
Function computes mesh at assembly level from list
of Part objects. The returned dictionary contains
parts with renumbered nodes and elements
\begin{quote}\begin{description}
\item[{Returns}] \leavevmode
dictionary of parts

\item[{Return type}] \leavevmode
defaultdict(\sphinxhref{https://docs.python.org/3.8/library/stdtypes.html\#list}{list})

\end{description}\end{quote}

\end{fulllineitems}

\index{exportToCalculix() (cofea.Mesh method)@\spxentry{exportToCalculix()}\spxextra{cofea.Mesh method}}

\begin{fulllineitems}
\phantomsection\label{\detokenize{pyCofea/api:cofea.Mesh.exportToCalculix}}\pysiglinewithargsret{\sphinxbfcode{\sphinxupquote{exportToCalculix}}}{\emph{\DUrole{n}{exportedFilename}}}{}
Function to export mesh to Calculix format
\begin{quote}\begin{description}
\item[{Parameters}] \leavevmode
\sphinxstyleliteralstrong{\sphinxupquote{exportedFilename}} (\sphinxhref{https://docs.python.org/3.8/library/stdtypes.html\#str}{\sphinxstyleliteralemphasis{\sphinxupquote{str}}}) \textendash{} name of the file to export mesh (eg ‘calculix.inp’)

\end{description}\end{quote}

\end{fulllineitems}

\index{exportToUnvFormat() (cofea.Mesh method)@\spxentry{exportToUnvFormat()}\spxextra{cofea.Mesh method}}

\begin{fulllineitems}
\phantomsection\label{\detokenize{pyCofea/api:cofea.Mesh.exportToUnvFormat}}\pysiglinewithargsret{\sphinxbfcode{\sphinxupquote{exportToUnvFormat}}}{\emph{\DUrole{n}{exportedFilename}}}{}
Function to export to UNV format
\begin{quote}\begin{description}
\item[{Parameters}] \leavevmode
\sphinxstyleliteralstrong{\sphinxupquote{exportedFilename}} (\sphinxhref{https://docs.python.org/3.8/library/stdtypes.html\#str}{\sphinxstyleliteralemphasis{\sphinxupquote{str}}}) \textendash{} name of the file to export mesh (eg ‘salome.unv’)

\end{description}\end{quote}

\end{fulllineitems}

\index{importFromAbaqusCae() (cofea.Mesh class method)@\spxentry{importFromAbaqusCae()}\spxextra{cofea.Mesh class method}}

\begin{fulllineitems}
\phantomsection\label{\detokenize{pyCofea/api:cofea.Mesh.importFromAbaqusCae}}\pysiglinewithargsret{\sphinxbfcode{\sphinxupquote{classmethod }}\sphinxbfcode{\sphinxupquote{importFromAbaqusCae}}}{\emph{\DUrole{n}{abqModelName}}, \emph{\DUrole{n}{abqPartList}}}{}
Function to initiate the Mesh object
\begin{quote}\begin{description}
\item[{Parameters}] \leavevmode\begin{itemize}
\item {} 
\sphinxstyleliteralstrong{\sphinxupquote{abqModelName}} (\sphinxhref{https://docs.python.org/3.8/library/stdtypes.html\#str}{\sphinxstyleliteralemphasis{\sphinxupquote{str}}}) \textendash{} name of the Abaqus model

\item {} 
\sphinxstyleliteralstrong{\sphinxupquote{abqPartList}} (\sphinxhref{https://docs.python.org/3.8/library/stdtypes.html\#list}{\sphinxstyleliteralemphasis{\sphinxupquote{list}}}) \textendash{} list of Abaqus Parts to export mesh from

\end{itemize}

\item[{Returns}] \leavevmode
object containing model mesh definition

\item[{Return type}] \leavevmode
{\hyperref[\detokenize{pyCofea/api:cofea.Mesh}]{\sphinxcrossref{Mesh}}}

\end{description}\end{quote}

\end{fulllineitems}

\index{importFromDbFile() (cofea.Mesh class method)@\spxentry{importFromDbFile()}\spxextra{cofea.Mesh class method}}

\begin{fulllineitems}
\phantomsection\label{\detokenize{pyCofea/api:cofea.Mesh.importFromDbFile}}\pysiglinewithargsret{\sphinxbfcode{\sphinxupquote{classmethod }}\sphinxbfcode{\sphinxupquote{importFromDbFile}}}{\emph{\DUrole{n}{pathToDbFile}}}{}
Function to initiate the Mesh object
\begin{quote}\begin{description}
\item[{Parameters}] \leavevmode
\sphinxstyleliteralstrong{\sphinxupquote{pathToDbFile}} (\sphinxhref{https://docs.python.org/3.8/library/stdtypes.html\#str}{\sphinxstyleliteralemphasis{\sphinxupquote{str}}}) \textendash{} path to the db file

\item[{Returns}] \leavevmode
object containing model mesh definition

\item[{Return type}] \leavevmode
{\hyperref[\detokenize{pyCofea/api:cofea.Mesh}]{\sphinxcrossref{Mesh}}}

\end{description}\end{quote}

\end{fulllineitems}

\index{saveToDbFile() (cofea.Mesh method)@\spxentry{saveToDbFile()}\spxextra{cofea.Mesh method}}

\begin{fulllineitems}
\phantomsection\label{\detokenize{pyCofea/api:cofea.Mesh.saveToDbFile}}\pysiglinewithargsret{\sphinxbfcode{\sphinxupquote{saveToDbFile}}}{\emph{\DUrole{n}{nameOfDbFile}}}{}
Function to save model data to db file
\begin{quote}\begin{description}
\item[{Parameters}] \leavevmode
\sphinxstyleliteralstrong{\sphinxupquote{nameOfDbFile}} (\sphinxhref{https://docs.python.org/3.8/library/stdtypes.html\#str}{\sphinxstyleliteralemphasis{\sphinxupquote{str}}}) \textendash{} name of the file that will be used to store data

\end{description}\end{quote}

\end{fulllineitems}


\end{fulllineitems}

\index{PartMesh (class in cofea)@\spxentry{PartMesh}\spxextra{class in cofea}}

\begin{fulllineitems}
\phantomsection\label{\detokenize{pyCofea/api:cofea.PartMesh}}\pysiglinewithargsret{\sphinxbfcode{\sphinxupquote{class }}\sphinxcode{\sphinxupquote{cofea.}}\sphinxbfcode{\sphinxupquote{PartMesh}}}{\emph{\DUrole{n}{partName}}, \emph{\DUrole{n}{partNodes}}, \emph{\DUrole{n}{partElements}}, \emph{\DUrole{n}{elementSets}\DUrole{o}{=}\DUrole{default_value}{None}}, \emph{\DUrole{n}{nodeSets}\DUrole{o}{=}\DUrole{default_value}{None}}}{}
PartMesh is an object used to store mesh properties for a given part.
\index{name (cofea.PartMesh attribute)@\spxentry{name}\spxextra{cofea.PartMesh attribute}}

\begin{fulllineitems}
\phantomsection\label{\detokenize{pyCofea/api:cofea.PartMesh.name}}\pysigline{\sphinxbfcode{\sphinxupquote{name}}}
name of the part
\begin{quote}\begin{description}
\item[{Type}] \leavevmode
\sphinxhref{https://docs.python.org/3.8/library/stdtypes.html\#str}{str}

\end{description}\end{quote}

\end{fulllineitems}

\index{nodes (cofea.PartMesh attribute)@\spxentry{nodes}\spxextra{cofea.PartMesh attribute}}

\begin{fulllineitems}
\phantomsection\label{\detokenize{pyCofea/api:cofea.PartMesh.nodes}}\pysigline{\sphinxbfcode{\sphinxupquote{nodes}}}
list of the Node objects
\begin{quote}\begin{description}
\item[{Type}] \leavevmode
list of Node objects

\end{description}\end{quote}

\end{fulllineitems}

\index{elements (cofea.PartMesh attribute)@\spxentry{elements}\spxextra{cofea.PartMesh attribute}}

\begin{fulllineitems}
\phantomsection\label{\detokenize{pyCofea/api:cofea.PartMesh.elements}}\pysigline{\sphinxbfcode{\sphinxupquote{elements}}}
list of the Element objects
\begin{quote}\begin{description}
\item[{Type}] \leavevmode
\sphinxhref{https://docs.python.org/3.8/library/stdtypes.html\#list}{list}

\end{description}\end{quote}

\end{fulllineitems}

\index{elementsByType (cofea.PartMesh attribute)@\spxentry{elementsByType}\spxextra{cofea.PartMesh attribute}}

\begin{fulllineitems}
\phantomsection\label{\detokenize{pyCofea/api:cofea.PartMesh.elementsByType}}\pysigline{\sphinxbfcode{\sphinxupquote{elementsByType}}}
dictionary of elements grouped by element type
\begin{quote}\begin{description}
\item[{Type}] \leavevmode
defaultdict(\sphinxhref{https://docs.python.org/3.8/library/stdtypes.html\#list}{list})

\end{description}\end{quote}

\end{fulllineitems}

\index{elSet (cofea.PartMesh attribute)@\spxentry{elSet}\spxextra{cofea.PartMesh attribute}}

\begin{fulllineitems}
\phantomsection\label{\detokenize{pyCofea/api:cofea.PartMesh.elSet}}\pysigline{\sphinxbfcode{\sphinxupquote{elSet}}}
dictionary of element sets grouped by element set name
\begin{quote}\begin{description}
\item[{Type}] \leavevmode
defaultdict(\sphinxhref{https://docs.python.org/3.8/library/stdtypes.html\#list}{list})

\end{description}\end{quote}

\end{fulllineitems}

\index{nSet (cofea.PartMesh attribute)@\spxentry{nSet}\spxextra{cofea.PartMesh attribute}}

\begin{fulllineitems}
\phantomsection\label{\detokenize{pyCofea/api:cofea.PartMesh.nSet}}\pysigline{\sphinxbfcode{\sphinxupquote{nSet}}}
dictionary of node sets grouped by node set name
\begin{quote}\begin{description}
\item[{Type}] \leavevmode
defaultdict(\sphinxhref{https://docs.python.org/3.8/library/stdtypes.html\#list}{list})

\end{description}\end{quote}

\end{fulllineitems}

\index{fromAbaqusCae() (cofea.PartMesh class method)@\spxentry{fromAbaqusCae()}\spxextra{cofea.PartMesh class method}}

\begin{fulllineitems}
\phantomsection\label{\detokenize{pyCofea/api:cofea.PartMesh.fromAbaqusCae}}\pysiglinewithargsret{\sphinxbfcode{\sphinxupquote{classmethod }}\sphinxbfcode{\sphinxupquote{fromAbaqusCae}}}{\emph{\DUrole{n}{abqPart}}}{}
Function to initiate object from Abaqus part.
\begin{quote}\begin{description}
\item[{Parameters}] \leavevmode
\sphinxstyleliteralstrong{\sphinxupquote{abqPart}} (\sphinxstyleliteralemphasis{\sphinxupquote{abqPart}}) \textendash{} Abaqus part object

\item[{Returns}] \leavevmode
object containing part mesh definition

\item[{Return type}] \leavevmode
{\hyperref[\detokenize{pyCofea/api:cofea.PartMesh}]{\sphinxcrossref{PartMesh}}}

\end{description}\end{quote}

\end{fulllineitems}

\index{getDiffFormatForElType() (cofea.PartMesh method)@\spxentry{getDiffFormatForElType()}\spxextra{cofea.PartMesh method}}

\begin{fulllineitems}
\phantomsection\label{\detokenize{pyCofea/api:cofea.PartMesh.getDiffFormatForElType}}\pysiglinewithargsret{\sphinxbfcode{\sphinxupquote{getDiffFormatForElType}}}{\emph{\DUrole{n}{oldElType}}, \emph{\DUrole{n}{newMeshFormat}}}{}
Function to retrieve name of element type for a
specific software
\begin{quote}\begin{description}
\item[{Parameters}] \leavevmode\begin{itemize}
\item {} 
\sphinxstyleliteralstrong{\sphinxupquote{oldElType}} (\sphinxhref{https://docs.python.org/3.8/library/stdtypes.html\#str}{\sphinxstyleliteralemphasis{\sphinxupquote{str}}}) \textendash{} name of the element type that needs to be
converted (eg ‘C3D8R’)

\item {} 
\sphinxstyleliteralstrong{\sphinxupquote{newMeshFormat}} (\sphinxhref{https://docs.python.org/3.8/library/stdtypes.html\#str}{\sphinxstyleliteralemphasis{\sphinxupquote{str}}}) \textendash{} new mesh format (eg ‘UNV’)

\end{itemize}

\item[{Returns}] \leavevmode
name of the element type in new format

\item[{Return type}] \leavevmode
\sphinxhref{https://docs.python.org/3.8/library/stdtypes.html\#str}{str}

\item[{Raises}] \leavevmode
\sphinxhref{https://docs.python.org/3.8/library/exceptions.html\#ValueError}{\sphinxstyleliteralstrong{\sphinxupquote{ValueError}}} \textendash{} error appears when the specific element type
    is not implemented

\end{description}\end{quote}

\end{fulllineitems}

\index{getElementsByType() (cofea.PartMesh method)@\spxentry{getElementsByType()}\spxextra{cofea.PartMesh method}}

\begin{fulllineitems}
\phantomsection\label{\detokenize{pyCofea/api:cofea.PartMesh.getElementsByType}}\pysiglinewithargsret{\sphinxbfcode{\sphinxupquote{getElementsByType}}}{}{}
Function used to create a dictionary of elements with
element types used as keys
\begin{quote}\begin{description}
\item[{Returns}] \leavevmode
dictionary of elements with elementtypes used as keys

\item[{Return type}] \leavevmode
defaultdict(\sphinxhref{https://docs.python.org/3.8/library/stdtypes.html\#list}{list})

\end{description}\end{quote}

\end{fulllineitems}

\index{getElementsFromLabelList() (cofea.PartMesh method)@\spxentry{getElementsFromLabelList()}\spxextra{cofea.PartMesh method}}

\begin{fulllineitems}
\phantomsection\label{\detokenize{pyCofea/api:cofea.PartMesh.getElementsFromLabelList}}\pysiglinewithargsret{\sphinxbfcode{\sphinxupquote{getElementsFromLabelList}}}{\emph{\DUrole{n}{labelList}}}{}
Function to retreive the elements from list of
elements labels.
\begin{quote}\begin{description}
\item[{Parameters}] \leavevmode
\sphinxstyleliteralstrong{\sphinxupquote{labelList}} (\sphinxhref{https://docs.python.org/3.8/library/stdtypes.html\#list}{\sphinxstyleliteralemphasis{\sphinxupquote{list}}}) \textendash{} list of integers referring to element labels

\item[{Returns}] \leavevmode
list of Element objects

\item[{Return type}] \leavevmode
\sphinxhref{https://docs.python.org/3.8/library/stdtypes.html\#list}{list}

\end{description}\end{quote}

\end{fulllineitems}

\index{getNodesFromLabelList() (cofea.PartMesh method)@\spxentry{getNodesFromLabelList()}\spxextra{cofea.PartMesh method}}

\begin{fulllineitems}
\phantomsection\label{\detokenize{pyCofea/api:cofea.PartMesh.getNodesFromLabelList}}\pysiglinewithargsret{\sphinxbfcode{\sphinxupquote{getNodesFromLabelList}}}{\emph{\DUrole{n}{labelList}}}{}
Function to retrieve the nodes from list of node labels
\begin{quote}\begin{description}
\item[{Parameters}] \leavevmode
\sphinxstyleliteralstrong{\sphinxupquote{labelList}} (\sphinxhref{https://docs.python.org/3.8/library/stdtypes.html\#list}{\sphinxstyleliteralemphasis{\sphinxupquote{list}}}) \textendash{} list of integers describing node labes

\item[{Returns}] \leavevmode
list of Node objects

\item[{Return type}] \leavevmode
\sphinxhref{https://docs.python.org/3.8/library/stdtypes.html\#list}{list}

\end{description}\end{quote}

\end{fulllineitems}

\index{reorderNodesInElType() (cofea.PartMesh method)@\spxentry{reorderNodesInElType()}\spxextra{cofea.PartMesh method}}

\begin{fulllineitems}
\phantomsection\label{\detokenize{pyCofea/api:cofea.PartMesh.reorderNodesInElType}}\pysiglinewithargsret{\sphinxbfcode{\sphinxupquote{reorderNodesInElType}}}{\emph{\DUrole{n}{meshFormat}}}{}
Function used to reoder nodes. By default the node
order is the same as in Abaqus.
\begin{quote}\begin{description}
\item[{Parameters}] \leavevmode
\sphinxstyleliteralstrong{\sphinxupquote{meshFormat}} (\sphinxhref{https://docs.python.org/3.8/library/stdtypes.html\#str}{\sphinxstyleliteralemphasis{\sphinxupquote{str}}}) \textendash{} Format of mesh to be converted to

\end{description}\end{quote}

\end{fulllineitems}

\index{setElementTypeFormat() (cofea.PartMesh method)@\spxentry{setElementTypeFormat()}\spxextra{cofea.PartMesh method}}

\begin{fulllineitems}
\phantomsection\label{\detokenize{pyCofea/api:cofea.PartMesh.setElementTypeFormat}}\pysiglinewithargsret{\sphinxbfcode{\sphinxupquote{setElementTypeFormat}}}{\emph{\DUrole{n}{newFormat}}}{}
Function to change element types in dictionary
elementsByType (for example from C3D20R to )
\begin{quote}\begin{description}
\item[{Parameters}] \leavevmode
\sphinxstyleliteralstrong{\sphinxupquote{newFormat}} (\sphinxhref{https://docs.python.org/3.8/library/stdtypes.html\#str}{\sphinxstyleliteralemphasis{\sphinxupquote{str}}}) \textendash{} new format (eg ‘UNV’)

\end{description}\end{quote}

\end{fulllineitems}


\end{fulllineitems}

\index{Element (class in cofea)@\spxentry{Element}\spxextra{class in cofea}}

\begin{fulllineitems}
\phantomsection\label{\detokenize{pyCofea/api:cofea.Element}}\pysiglinewithargsret{\sphinxbfcode{\sphinxupquote{class }}\sphinxcode{\sphinxupquote{cofea.}}\sphinxbfcode{\sphinxupquote{Element}}}{\emph{\DUrole{n}{elType}}, \emph{\DUrole{n}{elLabel}}, \emph{\DUrole{n}{elConnect}}, \emph{\DUrole{n}{partAllNodes}}}{}
Class used to store information about elements,
theirs type, label and connection to nodes
\index{type (cofea.Element attribute)@\spxentry{type}\spxextra{cofea.Element attribute}}

\begin{fulllineitems}
\phantomsection\label{\detokenize{pyCofea/api:cofea.Element.type}}\pysigline{\sphinxbfcode{\sphinxupquote{type}}}
element type (eg C3D4)
\begin{quote}\begin{description}
\item[{Type}] \leavevmode
\sphinxhref{https://docs.python.org/3.8/library/stdtypes.html\#str}{str}

\end{description}\end{quote}

\end{fulllineitems}

\index{label (cofea.Element attribute)@\spxentry{label}\spxextra{cofea.Element attribute}}

\begin{fulllineitems}
\phantomsection\label{\detokenize{pyCofea/api:cofea.Element.label}}\pysigline{\sphinxbfcode{\sphinxupquote{label}}}
element number
\begin{quote}\begin{description}
\item[{Type}] \leavevmode
\sphinxhref{https://docs.python.org/3.8/library/functions.html\#int}{int}

\end{description}\end{quote}

\end{fulllineitems}

\index{connectivity (cofea.Element attribute)@\spxentry{connectivity}\spxextra{cofea.Element attribute}}

\begin{fulllineitems}
\phantomsection\label{\detokenize{pyCofea/api:cofea.Element.connectivity}}\pysigline{\sphinxbfcode{\sphinxupquote{connectivity}}}
list of nodes that element is based on
\begin{quote}\begin{description}
\item[{Type}] \leavevmode
\sphinxhref{https://docs.python.org/3.8/library/stdtypes.html\#list}{list}

\end{description}\end{quote}

\end{fulllineitems}

\index{getNodeLabels() (cofea.Element method)@\spxentry{getNodeLabels()}\spxextra{cofea.Element method}}

\begin{fulllineitems}
\phantomsection\label{\detokenize{pyCofea/api:cofea.Element.getNodeLabels}}\pysiglinewithargsret{\sphinxbfcode{\sphinxupquote{getNodeLabels}}}{}{}
Function returns labels of nodes which building the element
\begin{quote}\begin{description}
\item[{Returns}] \leavevmode
list of integers describing node labels

\item[{Return type}] \leavevmode
\sphinxhref{https://docs.python.org/3.8/library/stdtypes.html\#list}{list}

\end{description}\end{quote}

\end{fulllineitems}


\end{fulllineitems}

\index{Node (class in cofea)@\spxentry{Node}\spxextra{class in cofea}}

\begin{fulllineitems}
\phantomsection\label{\detokenize{pyCofea/api:cofea.Node}}\pysiglinewithargsret{\sphinxbfcode{\sphinxupquote{class }}\sphinxcode{\sphinxupquote{cofea.}}\sphinxbfcode{\sphinxupquote{Node}}}{\emph{\DUrole{n}{nLabel}}, \emph{\DUrole{n}{nCoords}}}{}
Used to create node objects from external data
\index{label (cofea.Node attribute)@\spxentry{label}\spxextra{cofea.Node attribute}}

\begin{fulllineitems}
\phantomsection\label{\detokenize{pyCofea/api:cofea.Node.label}}\pysigline{\sphinxbfcode{\sphinxupquote{label}}}
Node number
\begin{quote}\begin{description}
\item[{Type}] \leavevmode
\sphinxhref{https://docs.python.org/3.8/library/functions.html\#int}{int}

\end{description}\end{quote}

\end{fulllineitems}

\index{coordinates (cofea.Node attribute)@\spxentry{coordinates}\spxextra{cofea.Node attribute}}

\begin{fulllineitems}
\phantomsection\label{\detokenize{pyCofea/api:cofea.Node.coordinates}}\pysigline{\sphinxbfcode{\sphinxupquote{coordinates}}}
Node coordinates
\begin{quote}\begin{description}
\item[{Type}] \leavevmode
\sphinxhref{https://docs.python.org/3.8/library/stdtypes.html\#tuple}{tuple}

\end{description}\end{quote}

\end{fulllineitems}

\index{changeLabel() (cofea.Node method)@\spxentry{changeLabel()}\spxextra{cofea.Node method}}

\begin{fulllineitems}
\phantomsection\label{\detokenize{pyCofea/api:cofea.Node.changeLabel}}\pysiglinewithargsret{\sphinxbfcode{\sphinxupquote{changeLabel}}}{\emph{\DUrole{n}{newLabel}}}{}
A function to change node label
\begin{quote}\begin{description}
\item[{Parameters}] \leavevmode
\sphinxstyleliteralstrong{\sphinxupquote{newLabel}} (\sphinxhref{https://docs.python.org/3.8/library/functions.html\#int}{\sphinxstyleliteralemphasis{\sphinxupquote{int}}}) \textendash{} new node number

\end{description}\end{quote}

\end{fulllineitems}


\end{fulllineitems}


Here’s a sample post list:
\begin{itemize}
\end{itemize}







\renewcommand{\indexname}{Index}
\printindex
\end{document}